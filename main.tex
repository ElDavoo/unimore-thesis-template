% https://texblog.org/2013/02/13/latex-documentclass-options-illustrated/
\documentclass[a4paper,12pt,draft,onecolumn,twoside,openright,notitlepage]{book}

\usepackage[italian,english]{babel}

% Margini, presi dal template CRIS
\usepackage[inner=3 cm,outer=2.5 cm,top=3 cm,bottom=2.5 cm]{geometry}

% Times New Roman
\usepackage{times}

% Abilita il supporto alle immagini
\usepackage{graphicx}
%Path relative to the main .tex file 
\graphicspath{ {./images/} }

% Mette i counter della pagina con sezione e sottolineatura, imposta anche stile
\usepackage{fancyhdr}
\fancypagestyle{mainmatter}{%		
		\fancyhf{} 
		\fancyhead{}
		\fancyhead[LE,RO]{\thepage}
		\fancyhead[LO]{\footnotesize{\leftmark}}
		\fancyhead[RE]{\footnotesize{\rightmark}}
		\fancyfoot{}
		\addtolength{\headwidth}{\marginparsep}
		\addtolength{\headheight}{2.5pt}
		\renewcommand{\headrulewidth}{0.3pt}
		\renewcommand{\footrulewidth}{0.0pt}
		}
\fancypagestyle{frontmatter}{%
		\fancyhf{} 
		\fancyhead[LE]{\footnotesize{\MakeUppercase{\thepage}}}
		\fancyhead[RO]{\footnotesize{\MakeUppercase{\thepage}}}
		\fancyhead[RE,LO]{}
		\fancyfoot{}
		\addtolength{\headwidth}{\marginparsep}
		\addtolength{\headheight}{2.5pt}
		\renewcommand{\headrulewidth}{0.0pt}
		\renewcommand{\footrulewidth}{0.0pt}
		}
\pagestyle{fancy}
		\fancyhf{} 
		\fancyhead{}
		\fancyhead[LE,RO]{\thepage} 
		\fancyhead[LO]{\footnotesize{\leftmark}}
		\fancyhead[RE]{\footnotesize{\rightmark}}
		\fancyfoot{}
		\addtolength{\headwidth}{\marginparsep}
		\addtolength{\headheight}{2.5pt}
		\renewcommand{\headrulewidth}{0.3pt}
		\renewcommand{\footrulewidth}{0.0pt}

% Li toglie nelle pagine vuote
\usepackage{emptypage}

% Dà un po' di spazio in più alle parole in caso non ci entrino nella riga
\setlength{\emergencystretch}{4em}


% environment for bash code
\usepackage{listings}
\lstset{ %
  language=bash,                % the language of the code
  basicstyle=\footnotesize,           % the size of the fonts that are used for the code
  numbers=left,                   % where to put the line-numbers
  numberstyle=\footnotesize,          % the size of the fonts that are used for the line-numbers
  stepnumber=1,                   % the step between two line-numbers. If it's 1, each line 
                                  % will be numbered
  numbersep=5pt,                  % how far the line-numbers are from the code
  backgroundcolor=\color{white},      % choose the background color. You must add \usepackage{color}
  showspaces=false,               % show spaces adding particular underscores
  showstringspaces=false,         % underline spaces within strings
  showtabs=false,                 % show tabs within strings adding particular underscores
%  frame=single,                   % adds a frame around the code
  rulecolor=\color{black},        % if not set, the frame-color may be changed on line-breaks within not-black text (e.g. commens (green here))
  tabsize=2,                      % sets default tabsize to 2 spaces
  captionpos=b,                   % sets the caption-position to bottom
  breaklines=true,                % sets automatic line breaking
  breakatwhitespace=false,        % sets if automatic breaks should only happen at whitespace
  title=\lstname,                   % show the filename of files included with \lstinputlisting;
                                  % also try caption instead of title
  numberstyle=\tiny\color{gray},        % line number style
  keywordstyle=\textbf,          % keyword style
  commentstyle=\color{darkgreen},       % comment style
%  stringstyle=\color{mauve},         % string literal style
  escapeinside={\%*}{*)},            % if you want to add a comment within your code
  morekeywords={*,...,insert,-}               % if you want to add more keywords to the setù
}

% environment for python code
\lstset{
language=Python,
breaklines=true,
breakatwhitespace=true ,
backgroundcolor=\color{light-gray}
}

% Interlinea
\linespread{1.5}
% Un po' più piccola per gli elenchi numerati
\usepackage{enumitem}
\setlist[enumerate]{itemsep = -1 mm}

%Consente di usare \code al posto di \texttt 
\newcommand{\code}{\texttt}

%Glossario
\usepackage[acronym,toc]{glossaries}
\makeglossaries

%Usalo per aggiungere un capitolo al TOC senza numerarlo
\newcommand*{\nchapter}[1]{
    \chapter*{#1}
    \addcontentsline{toc}{chapter}{#1}
}

% Supporto ai line break negli URL
\usepackage{xurl}

% Rende cliccabili i link
% (Overleaf dice che questo è l'ultimo pacchetto da caricare)
% NOTA: I link sono disattivati se il documento è in draft!
\usepackage{hyperref}

\hypersetup{
	pdfpagemode=FullScreen,
	% FIXME Deduplicare questi due dati
	pdftitle={Titolo tesi},
	pdfauthor={Autore},
}

\begin{document}

% Disattiva la numerazione
\pagenumbering{gobble}

\newacronym{gcd}{GCD}{Greatest Comjffdyhmon Divisor}
\newglossaryentry{divooneh}
{
        name=divooneh,
        description={Someone who is a little bit crazy}
}
\newglossaryentry{aghel}
{
        name=aghel,
        description={Someone who is clever and serious}
}
%\glsaddallunused
\printglossary[type=\acronymtype]
\printglossary
\newpage


\thispagestyle{plain}

\begin{titlepage}
    \begin{center}

		Università degli Studi di Modena e Reggio Emilia\\
		\large
		Dipartimento di Ingegneria "Enzo Ferrari"

        \vspace{2.5cm}
        
		\Large
		Corso di laurea Magistrale in Ingegneria Informatica\\
		
		\large
		Percorso "Cloud and Cyber Security"		
		
		\vspace{2.5cm}
        
        \Huge
        \textbf{Titolo tesi}

        \vspace{3cm}

		\large
		

		\begin{flushright}
		Prova finale di:\\
		Nome Cognome
		\end{flushright}		        
        
        
		\Large        
        \vspace{3cm}
        
        \begin{flushleft}
        Relatore:\\
        A B
        \end{flushleft}
                
        \begin{flushleft}
        Relatore2:\\
        C D
        \end{flushleft}
            
		\vfill
		
		\small
		Anno Accademico 2023/2024
            
    \end{center}
\end{titlepage}

\begin{center}

\mbox{}
\vfill

\begin{flushright}
\textit{Ai miei genitori}
\end{flushright}

\newpage

\end{center}


\thispagestyle{plain}

\begin{otherlanguage}{italian}
\section*{Sommario}
Questo è un template per una tesi (magistrale o triennale) a UniMORE. Ciao ciaoc ac ca s das dsa d sa da sdas das da d as d 
\newpage
\end{otherlanguage}

\thispagestyle{plain}
\section*{Abstract}
This is a thesis template for UniPOG.

\newpage

\pagestyle{frontmatter}
% Resetta e imposta la numerazione a caratteri romani
\frontmatter

\tableofcontents

% Resetta e imposta la numerazione a numeri arabi
\mainmatter
\pagestyle{mainmatter}

\chapter{Introduzione}

\section{Hello World}
Ciao

\chapter{State of the art}

\section{Hello World}
Ciao\\
This paper \cite{kim_firmae_2020} is so cool, but I need tto do f6 + f11 + f6 + f6 + f7 to compile everything

\chapter{Methodology}

\section{Hello World}
Ciao

\chapter{Conclusion}

\section{Hello World}
Ciao

\begin{flushleft}
\bibliographystyle{plain}
\bibliography{references}{}
\end{flushleft}

\chapter*{Appendice}
appendix
\newpage

\printglossary[type=\acronymtype]
\printglossary
\newpage

\section*{Ringraziamenti}
Ringrazio mamma, papà ed Emiliano ed Antonio e Solida e Luca e Luca e Feli e Fede etc etc

\end{document}
